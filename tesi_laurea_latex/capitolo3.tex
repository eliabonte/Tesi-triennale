\chapter{Conclusioni}
\label{cha:conclusioni}
Lo scopo di questa tesi è stato quello di approfondire l'implementazione in Simulink di un sistema dinamico e successivamente quello di ottimizzare, tramite l'algoritmo di ottimizzazione bayesiana, i parametri delle strategie di controllo.\\

In questo elaborato si è cercato di mettere in luce le differenze tra diverse strategie di controllo, analizzando soprattutto indice di performance e funzione di trasferimento, in modo da capire quale possa garantire delle prestazioni migliori in termini di comfort del passeggero e fornire quindi la soluzione migliore per il controllo di sospensioni semi-attive di tipo magnetoreologico.\\
La taratura dei parametri è stata fatta tramite l'utilizzo dell'algoritmo di ottimizzazione bayesiana, che ha fornito dei buoni risultati, ciò non esclude però che possano esserci delle tecniche migliori dell'ottimizzatore per trovare i valori corretti dei parametri delle leggi di controllo.\\

Lo svolgimento di questa tesi, mi ha dato l'opportunità di approfondire maggiormente, a livello pratico, temi legati al controllo di sistemi dinamici e la relativa implementazione, tramite software estremamente utilizzati in ambito ingegneristico.