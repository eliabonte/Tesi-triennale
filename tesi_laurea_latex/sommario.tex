\chapter*{Sommario} % senza numerazione
\label{sommario}

\addcontentsline{toc}{chapter}{Sommario} % da aggiungere comunque all'indice

Le sospensioni sono una delle parti fondamentali di un veicolo, costituiscono il collegamento principale tra la strada e il telaio e sono responsabili della sicurezza e del comfort del passeggero a bordo.\\
In questo lavoro sono state analizzate, tramite simulazioni su un modello quarter car, le prestazioni di comfort di diverse strategie di controllo per le sospensioni semi-attive di tipo magnetoreologico. La taratura dei parametri delle leggi di controllo è stata fatta tramite l'algoritmo di ottimizzazione bayesiana.\\
L'obiettivo iniziale era quello di osservare le prestazioni di due diverse leggi di controllo, lineare e quadratica, analizzarle in confronto a una configurazione passiva, e capire se l'algoritmo di ottimizzazione utilizzato fosse una tecnica efficiente per la taratura dei parametri. Nel corso del lavoro sono emersi però interessanti risultati e per questo si è voluto implementare un'ulteriore strategia di controllo, in modo da realizzare un'analisi più completa possibile.\\

Per la stesura della tesi si è fatto riferimento principalmente a \cite{controlMRdampers}.\\

Nella realizzazione di questa tesi sono stati utilizzati in parallelo i software MATLAB e Simulink. Quest'ultimo necessario per la realizzazione dello schema a blocchi del modello quarter car e per le simulazioni del sistema dinamico, mentre con l'ausilio di MATLAB sono stati elaborati i parametri necessari per il sistema dinamico, è stato implementato l'algoritmo di ottimizzazione e sono stati raccolti i dati delle simulazioni.\\



